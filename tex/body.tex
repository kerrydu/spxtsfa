% readme.tex -- a short example of how each Stata Journal insert should be
% organized.

\inserttype{article}

\author{Short article author list}{%
	Kerui Du\\
	School of Management\\
	Xiamen University\\
	Xiamen, China\\
	kerrydu@xmu.edu.cn \\
	\\
	
	\and
	Luis Orea \\
	Department of Economics \\
	School of Economics and Business\\
	University of Oviedo \\
	Oviedo, Spain \\
	lorea@uniovi.es \\
	\\
	
	\and
	Inmaculada C. Álvarez \\
	Department of Economics \\
	Universidad Autónoma de Madrid \\
	Madrid, Spain\\
	inmaculada.alvarez@uam.es
  
  
}

\title[Short toc article title]{Fitting spatial stochastic frontier models in Stata}
\maketitle


\begin{abstract}
In this article, we introduce a new command spxtsfa for fitting spatial stochastic frontier models in Stata. Over the last decades, An important theoretical progress of stochastic frontier models is the incorporation of various types of spatial components. Models with the ability to account for spatial dependence and spillovers have been developed for efficiency and productivity analysis, drawing extensive attention from industry and academia. Due to the unavailability of the statistical packages, the empirical applications of the new stochastic frontier models appear to be lagging. The spxtsfa command provides a routine for estimating the spatial stochastic frontier models in the style of \cite{orea2019new} and \cite{galli2022spatial}, enabling users to handle different sources of spatial dependence. In the presented article, we introduce the spatial stochastic frontier models, describe the syntax and options of the new command, and provide several examples to illustrate its usage.
	
	\keywords{stochastic frontier models, SFA, spatial dependence, technical efficiency, spillovers }
\end{abstract}


% discussion of the Stata Journal document class.
%\input sj.tex
% discussion of the Stata Press LaTeX package for Stata output.
%\input stata.tex

\section{Introduction}\label{sec_intro}
%[background]


Producers might fail in optimizing their production activities, causing deviation from the maximum output or the minimum cost. Economic researchers proposed the concept of technical efficiency, which measures how well a producer is utilizing its resources to produce goods or services. A technically efficient organization makes the maximum outputs given the amount of inputs or uses the minimum amount of inputs to produce a given level of output. On the contrary, technically inefficient organization produce fewer outputs given the same inputs or uses more inputs than necessary to produce the same output. Technical efficiency is important because it allows organizations or economies to achieve their goals with the least amount of resources possible, which can lead to cost savings and increased profitability. 

\cite{aignerFormulationEstimationStochastic1977} and \cite{meeusen1977efficiency} introduced stochastic frontier models for evaluating technical efficiency. The essential concept behind these models is to divide the observed output of a production process into two components, namely the "frontier" output, signifying the maximum feasible output, given the inputs utilized in the production process, and the "residual" output, denoting the production process's inefficiency. Following these initial works, stochastic frontier models gained extensive use as a tool for scrutinizing productivity and efficiency. 

Methodologically, econometricians have expanded the horizons of stochastic frontier models in various directions. To name a few, \cite{battese1995model} incorporated the determinants of inefficiency. \cite{wang2003stochastic} developed the stochastic frontier model with scaling properties to capture the shape of the distribution of inefficiency. \cite{greene2005fixed} extended the stochastic models with the random effects and the “true” fixed effects. \cite{belotti2018consistent}, \cite{chen2014consistent}, and \cite{ wang2010estimating} circumvented the "incidental parameters problem" in the fixed effects stochastic frontier model through model transformation. \cite{karakaplan2017handling} developed an endogenous stochastic frontier model to control for the endogeneity in the frontier or inefficiency. 


In recent years, stochastic frontier models have undergone further extension to account for spatial dependence and spatial spillover effects. \cite{glass2016spatial} constructed a spatial Durbin stochastic model considering both global and local spatial dependence. \cite{kutluSpatialStochasticFrontier2020} proposed a spatial stochastic frontier model with endogenous frontier and environmental variables. \cite{glass2016spatial} and \cite{kutluSpatialStochasticFrontier2020} combine the concepts of spatial econometrics and stochastic frontier analysis by including the spatial lag of the dependent variable. On the other hand, \cite{orea2019new} developed a new stochastic frontier model with spatial correlation in both noise and inefficiency terms. \cite{galli2022spatial} integrated the two different modeling ideas to specify four different sources of spatial dependence fully.  


With the increasing demand in the last decades to analyze technical efficiency, Stata provides official commands frontier and xtfrontier for cross-sectional and panel stochastic model estimation, respectively. \cite{belotti2013stochastic} developed sfcross and sfpanel commands accommodating more different distribution assumptions and allowing fixed-effect and random-effect models with the consideration of heteroscedasticity. \cite{karakaplan2017fitting} introduced the sfkk command for estimating endogenous stochastic frontier models. \cite{mustafaugurkarakaplan2018xtsfkk} supplemented the xtsfkk command for fitting the endogenous panel stochastic frontier model. \cite{kumbhakarpractitioner} provides a practitioner’s guide to stochastic frontier analysis with a suite of Stata commands (including sfmodel, sfpan, sf\_fixeff, and sfprim).

In this article, we introduce spxtsfa, a new command for fitting spatial stochastic frontier models in the style of \cite{orea2019new} and \cite{galli2022spatial}. The proposed spxtsfa command not only allows getting more accurate inefficiency scores \citep[see e.g.][]{orea2018spatial} but also examining relevant economic issues that a non-spatial stochastic frontier model tends to overlook. For instance, in microdata applications, the new command can be used to test whether the production/cost function can be viewed as a purely deterministic (engineering) process where the firm controls all the inputs \citep[see e.g.][]{druska2004generalized}. A distinctive feature of the spxtsfa command is that it allows estimating a stochastic frontier model with cross-sectional correlation in the inefficiency term, a specification that is useful in applications where some firms benefit from best practices implemented in adjacent firms due to, for instance, agglomeration economies, knowledge spillovers, technology diffusion or R\&D spillovers. This could especially be the case if (local) firms belong to communitarian networks (e.g. cooperatives) or common technicians (consultants) are advising all local firms. In practice, the proposed spxtsfa command can be useful to capture a kind of behavioral correlation, for instance when firms tend to “keep an eye” on the decisions of other peer firms trying to overcome the limitations caused by the lack of information or they simply emulate each other. It is finally germane to mention that the spxtsfa command also allows capturing cross-sectional effects that might be caused by non-spatial factors (e.g., the regulation environment) if we define appropriately the so-called weight (W) matrix. A proper definition of the W matrix might, for instance, allow us to examine the existence of knowledge spillovers from supplier and user firms. 

As \cite{orea2019new} point out, the proposed spxtsfa command can be implemented using macro-level data (e.g. data of countries, regions or industries) due to the abundant evidence of important feedback processes between neighboring or non-distant regions justify the use of SAR and Durbin frontier functions in macrodata applications. The spatial weight matrix specification commonly adopted in regional economics is based on geographical distance. However, as aforementioned, the weight matrix can be defined using a non-spatial criterion.  In this sense, \cite{liu2023industry} state that the mode of production in the world economy is characterized by the division of global value chains (GVCs) and, hence, the spatial weight matrix should be constructed using the economic distance between industries within/across national economies. In this case, the proposed spxtsfa command can be used to estimate spatial SAR and Durbin frontier functions in order to examine the diffusion of knowledge and technology among the participants in the international production network. It is also makes sense to estimate a stochastic frontier model with cross-sectional correlation in the inefficiency term using macrodata if we change the interpretation of the estimated correlation. In these applications, the spatial correlation in the inefficiency term likely captures barriers and distortions to the efficient allocation of resources across firms that are common to several regions, such as regulation, labor market trends or common institutions \citep[see e.g.][]{orea2023industry}.  


The remainder of this article unfolds as follows: Section 2 provides a brief description of the models in \cite{orea2019new} and \cite{galli2022spatial}; Section 3 explain the syntax and options of spxtsfa; Section 4 and 5 present simulated data examples to illustrate the usage of the command; and section 6 concludes the article.


\section{The model}\label{sec_method}
%[intro]
In this section, we briefly describe the spatial stochastic frontier models developed by \cite{orea2019new} and \cite{galli2022spatial}. The exposition here is only introductory. Please refer to the cited papers for more technical details.  

Based on the transposed version of \cite{wang2010estimating} model, \cite{orea2019new}  proposed a spatial stochastic frontier model which accommodates spatially-correlated inefficiency and noise terms. The model is formulated as in Eqs.\eqref{eq1}-\eqref{eq3}, for $i=1,...,N$ and $t=1,..,T$:

\begin{equation}\label{eq1}
 Y_{it} = X_{it}'\beta + \tilde{v}_{it}-s\tilde{u}_{it}
\end{equation}

\begin{equation}\label{eq2}
	\tilde{v}_{it} =v_{it}+ \gamma W_{i}^{vt}\tilde{v}_{.t} 
\end{equation}

\begin{equation}\label{eq3}
	\tilde{u}_{it} =u_{it}+ \tau W_{i}^{ut}\tilde{u}_{.t} 
\end{equation}

 Eq.\eqref{eq1}  describes the stochastic frontier function where $Y_{it}$ is the dependent variable and $X_{it}$ is a $k \times 1$ vector of variables shaping the frontier; $s=1$ for the production function and  $s=-1$ for the cost function; $\tilde{v}_{it}$ and $\tilde{u}_{it}$ represent  idiosyncratic noise and inefficiency, respectively. In  Eqs.\eqref{eq2} and \eqref{eq3}, $W_{i}^{vt}=(W_{i1}^{vt},...,W_{iN}^{vt})$ and $W_{i}^{vt}=(W_{i1}^{vt},...,W_{iN}^{vt})$ are two known $1 \times N$ cross-sectional weight vectors  depicting the structure of the  cross-sectional relationship for idiosyncratic noise and inefficiency terms, respectively; $\tilde{v}_{.t}=(\tilde{v}_{1t},...,\tilde{v}_{Nt})' $ and $\tilde{u}_{.t}=(\tilde{u}_{1t},...,\tilde{u}_{Nt})'$; $v_{it}$  is a random variable following the distribution $N(0,\sigma_v^2)$ and $u_{it}=h(Z_{it}'\delta)u_t^*$. $h(Z_{it}'\delta)$ is the scaling function where $Z_{it}$ is a $l \times 1$ vector of variables affecting individuals' inefficiency  and $u_t^*$ is a non-negative random variable following the distribution $N^+(0,\sigma_{u}^2)$.  Using matrix notation, we can rewrite Eqs.\eqref{eq2} and \eqref{eq3} as
 
 \begin{equation}\label{eq2b}
 	\tilde{v}_{.t} =(I_N-\gamma W^{vt})^{-1}v_{.t} 
 \end{equation}
 
 \begin{equation}\label{eq3b}
 		\tilde{u}_{.t} =(I_N-\tau W^{ut})^{-1}h(Z_{.t}\delta)u_t^* = \tilde{h}_{.t}u_t^*
 \end{equation}
 where $Z_{.t}=(Z_{1t},...,Z_{Nt})'$;$\tilde{h}_{.t}=(I_N-\tau W^{ut})^{-1}h(Z_{.t}\delta)$.
 
  The above model captures the spatial correlation of  the random error and inefficiency terms with the spatial autoregressive (SAR) process \footnote{\cite{orea2019new} also considered a specification of the spatial moving average process.}.  Referring to \cite{wang2010estimating}, we can obtain the following log-likelihood function for each period $t$:
  \begin{equation}\label{eq5}
 	\begin{aligned}
 		\ln L_{t}= & -\frac{N}{2} \ln (2 \pi)-\frac{1}{2} \ln |\Pi|-\frac{1}{2} \tilde{\varepsilon}_{.t} \Pi^{-1} \tilde{\varepsilon}_{.t} \\
 		& +\frac{1}{2}\left(\frac{\mu_{*}^{2}}{\sigma_{*}^{2}}\right)+\ln \left[\sigma_{*} \Phi\left(\frac{\mu_{*}}{\sigma_{*}}\right)\right]-\ln \left(\frac{1}{2}\sigma_{u} \right)
 	\end{aligned}
 \end{equation}
where $\Pi=\sigma_v^2(I_N-\rho W^{yt})^{-1}[(I_N-\rho W^{yt})^{-1}]'$; $ \tilde{\varepsilon}_{.t} = ( \tilde{\varepsilon}_{1t},..., \tilde{\varepsilon}_{Nt})', \tilde{\varepsilon}_{it}=s(Y_{it}-X_{it}' \beta)$, and 
\begin{equation}
	\mu_*  =\frac{-\tilde{\varepsilon}_{.t}^{\prime} \Pi^{-1} \tilde{h}_{.t}}{\tilde{h}_{.t}' \Pi^{-1} \tilde{h}_{.t}+1 / \sigma_u^2}
\end{equation}
\begin{equation}
	\sigma_*^2  =\frac{1}{\tilde{h}_{.t}^{\prime} \Pi^{-1} \tilde{h}_{.t}+1 / \sigma_u^2}
\end{equation}

 
\cite{galli2022spatial} further incorporated the spatial lags of the dependent variable and the input variables into \cite{orea2019new} model, which additionally measures global and local spatial spillovers affecting the frontier function.  The model is expressed as
\begin{equation}\label{gallimodel}
	Y_{it} = \rho W_{i}^{yt}Y_{.t}+X_{it}'\beta+ W_{i}^{xt}X_{.t} \theta + \tilde{v}_{it}+s\tilde{u}_{it}
\end{equation}
where $W_{i}^{yt}=(W_{i1}^{yt},...,W_{iN}^{yt})$ and $W_{i}^{xt}=(W_{i1}^{xt},...,W_{iN}^{xt})$ are two known $1 \times N$ cross-sectional weight vectors \footnote{We index $W_{i}^{yt}$, $W_{i}^{xt}$, $W_{i}^{ut}$, and $W_{i}^{vt}$ with superscript $yt$, $xt$, $ut$, and $vt$, respectively. This indicates the spatial weight matrix can be time-varying and different across various spatial components}; $Y_{.t} = (Y_{1t},..., Y_{Nt})'$; $X_{.t} = (X_{1t},..., X_{Nt})'$.  This model gives rise to the following log-likelihood function for each period $t$: 

  \begin{equation}\label{gallilik}
	\begin{aligned}
		\ln L_{t}= & ln|I_N - \rho W^{yt}|-\frac{N}{2} \ln (2 \pi)-\frac{1}{2} \ln |\Pi|-\frac{1}{2} \tilde{\varepsilon}_{.t} \Pi^{-1} \tilde{\varepsilon}_{.t} \\
		& +\frac{1}{2}\left(\frac{\mu_{*}^{2}}{\sigma_{*}^{2}}\right)+\ln \left[\sigma_{*} \Phi\left(\frac{\mu_{*}}{\sigma_{*}}\right)\right]-\ln \left(\frac{1}{2}\sigma_{u} \right)
	\end{aligned}
\end{equation}
where $ \tilde{\varepsilon}_{.t} = ( \tilde{\varepsilon}_{1t},..., \tilde{\varepsilon}_{Nt})', \tilde{\varepsilon}_{it}=s(Y_{it}-X_{it}' \beta - \rho W_{i}^{yt}Y_{.t} -W_{i}^{xt}X_{.t} \theta)$. 

Summing the time-specific log-likelihood  functions over all periods yields the overall likelihood function for the whole sample, i.e., $lnL=\sum_{t=1}^TlnL_{t}$. Then, numerically maximize the overall log-likelihood function to obtain consistent estimates of the parameters in the above models.  Specifically, we use Stata {\tt ml model} routine with the {\tt method-d0} evaluator to program the {\tt spxtsfa} command. Following \cite{gude2018heterogeneous}, we parameterize $\rho$, $\gamma$, and $\tau$ as Eq.\eqref{para} to ensure the standard regularity condition for the spatial autoregressive models.
\begin{equation}\label{para}
\begin{aligned}
	& \eta=\left(\frac{1}{r_{\text {min }}}\right)(1-p)+\left(\frac{1}{r_{\max }}\right) p \\
	& 0 \leq p=\frac{\exp \left(\delta_0\right)}{1+\exp \left(\delta_0\right)} \leq 1
\end{aligned}
\end{equation}
where $\eta$ stands for one of $\rho$, $\gamma$, and $\tau$;  $r_{\text {min }}$ and $r_{\text {max}}$ are respectively the minimum and maximum eigenvalues of the corresponding spatial weight matrix. 

In summary, \cite{galli2022spatial} provided a fully comprehensive specification of four different types of spatial dependence: global spillovers of dependent variable $Y_{it}$, local spillovers of input variables $X_{it}$, cross-sectional correlation of idiosyncratic noise  $v_{it}$ and inefficiency $u_{it}$. We term this full model "$yxuv$-SAR". Some restrictions can be imposed on the specific parameters to generate the following  models (summarized in Table \ref{Tab01}), which can be estimated by the {\tt spxtsfa} command.


% Please add the following required packages to your document preamble:
% \usepackage{booktabs}
\begin{table}[htbp]
	%\scriptsize
	
	\centering
	
	\caption{Specific  models with restricted parameters}
	
	\label{Tab01}
	\begin{tabular}{@{}llllllllllll@{}}
		\toprule
		 & $yuv$ & $xuv$ & $yv$ & $yu$ & $y$ & $xuv$ & $xv$ & $xu$ & $uv$ & $u$ & $v$  \\ \midrule
		$\rho$   &     & 0   &    &    &   & 0   & 0  & 0  & 0  & 0 & 0 \\
		$\theta$ & 0   &     & 0  & 0  & 0 &     &    &    & 0  & 0 & 0 \\
		$\gamma$ &     &     &    & 0  & 0 &     &    & 0  &    & 0 &   \\
		$\tau$   &     &     & 0  &    & 0 &     & 0  &    &    &   & 0 \\ \bottomrule
	\end{tabular}
\end{table}



\endinput

% discussion of the Stata Press LaTeX package for Stata output.

\section{The spxtsfa command}
{\tt spxtsfa} estimates spatial stochastic frontier models in the style of \cite{orea2019new} and \cite{galli2022spatial}.

\subsection{Syntax}

Estimation syntax

\begin{stsyntax}
	spxtsfa\
    \depvar\
    \optindepvars\,
	uhet(\varlist)
	\optional{
		noconstant
		cost
		wy({\it wyspec})
		wx({\it wxspec})
		wu({\it wuspec})
		wv({\it wvspec})
		normalize({\it norm\_method})
		wxvars(\varlist)
		\underbar{init}ial({\it matname})
		mlmodel({\it model\_options})
		mlsearch({\it search\_options})
		mlplot
		mlmax({\it maximize\_options})
		nolog
		mldisplay({\it display\_options})
		level(\num)
		lndetmc({\it numlist})
		te(\newvarname)
		genwxvars
		delmissing
		constraints(\it constraints)
	}
\end{stsyntax}



\noindent Version syntax

\begin{stsyntax}
	spxtsfa\
	, version
\end{stsyntax}


\noindent Replay syntax

\begin{stsyntax}
	spxtsfa\
	\optional{, level(\num) }
\end{stsyntax}

\subsection{Options}

\hangpara
{\tt uhet(\varlist)} specifies explanatory variables for technical inefficiency variance  function depending on a linear combination of \varlist. It is required.

\hangpara
{\tt noconstant} suppresses constant term.

\hangpara
{\tt cost} specifies the frontier as a cost function. By default, the production function is assumed.

\hangpara
{\tt wy({\it wyspec})} specifies the spatial weight matrix for lagged dependent variable. The expression is wy($W_1$ $ [W_2 ... W_T]$ [,{\it mata array}]).  By default, the weight matrices are {\tt Sp} objects. mata declares weight matrices are mata matrices. If one weight matrix is specified, it assumes a time-constant weight matrix. For time-varying cases, $T$ weight matrices should be specified in time order. Alternatively, using array to declare weight matrices are stored in an array.  If only one matrix is stored in the specified array, the time-constant weight matrix is assumed.  Otherwise, the keys of the array specify time information, and the values store time-specific weight matrices.

\hangpara
{\tt wx({\it wxspec})} specifies the spatial weight matrix for lagged independent variable. The expression is the same as {\tt wy({\it wyspec})}.

\hangpara
{\tt wu({\it wuspec})} specifies the spatial weight matrix for lagged independent variable. The expression is the same as {\tt wy({\it wyspec})}.

\hangpara
{\tt wv({\it wvspec})} specifies the spatial weight matrix for lagged independent variable. The expression is the same as {\tt wy({\it wyspec})}.

\hangpara
{\tt normalize({\it norm\_method})} specifies  one of the four available normalization techniques: row, col, minmax, and spectral.

\hangpara
{\tt wxvars(\varlist)} specifies spatially lagged independent variables.


\hangpara
{\tt \underbar{init}ial({\it matname})} specifies  the initial values of the estimated parameters with matrix {\it matname}.

\hangpara
{\tt mlmodel({\it model\_options})} specifies the  {\tt ml model} options.

\hangpara
{\tt mlsearch({\it search\_options})} specifies the  {\tt ml search} options.

\hangpara
{\tt mlplot} specifies using  {\tt ml plot} to search better initial values of spatial dependence parameters.

\hangpara
{\tt mlmax({\it maximize\_options})} specifies the  {\tt ml maximize} options.

\hangpara
{\tt nolog} suppresses the display of the criterion function iteration log.

\hangpara
{\tt mldisplay({\it display\_options})} specifies the  {\tt ml display} options.

\hangpara
{\tt level(\num)} sets confidence level; default is level(95).

\hangpara
{\tt lndetmc({\it numlist})} uses the trick of \cite{BARRY199941} to solve the inverse of $(I_N - \rho W)$.  The order of {\it numlist} is iterations, maxorder. {\tt lndetmc(50 100)} specifies that the number of iterations is 50 and the maximum order of moments is 100. 

\hangpara
{\tt te({\it newvarname})} specifies a new variable name to store the estimates of technical efficiency.

\hangpara
{\tt genwxvars} generates the spatial Durbin terms. It is activated only when {\tt wxvars(\varlist)} is specified.

\hangpara
{\tt delmissing} allows estimation  when missing values are present by  removing the corresponding units from spatial matrix. 

\hangpara
{\tt constraints(\it constraints)}  specifies linear constraints for the estimated model. 


\subsection{Dependency of spxtsfa}
{\tt spxtsfa} depends on the {\it moremata }package. If not already installed, you can install it by typing ssc install moremata.


%\section{Examples with simulated data}\label{sec_example}
\section{Examples}\label{sec_example}
In this section, we use simulated data to  exemplify the use of the \textit{spxtsfa} command.  Referring to , we first consider the $yxuv$-SAR model specified by the following data-generating process (DGP 1) with $i=1,...,300$ and $t=1,..,20$,

\begin{equation}\label{dgp1}
	Y_{it} = 0.3W_{i}Y_{.t}+2X_{it}+ 0.3W_{i}X_{.t}  + \tilde{v}_{it}-\tilde{u}_{it}
\end{equation}
where $\tilde{v}_{it}$ and $\tilde{u}_{it}$ are defined as in Eqs.\eqref{eq2} and \eqref{eq3} with $\gamma=0.3$, $\tau=0.3$, $\delta=2$, $\sigma_{u}^2=0.2$ and $\sigma_v^2 =0.2$. All the spatial matrices for the four spatial components are the same and time-invariant, created from a binary contiguity spatial weight matrix. We generate the exogenous variables $X_{it}$ and $Z_{it}$ from the standard normal distribution, respectively. With the sample generated by DGP 1, we can fit the model in the following syntax.

\begin{stlog}
	. use spxtsfa_DGP1.dta
{\smallskip}
. xtset id t 
{\smallskip}
Panel variable: id (strongly balanced)
 Time variable: t, 1 to 20
         Delta: 1 unit
{\smallskip}
. * importing spatial weight matrix from spxtsfa_wmat1.mmat
. mata mata matuse spxtsfa_wmat1.mmat,replace
(loading w1[300,300])
{\smallskip}
. * fitting the model
. spxtsfa y x, uhet(z) noconstant  wy(w1,mata) wx(w1,mata) wu(w1,mata) wv(w1,mata) wxvars(x) nolog
{\smallskip}
Spatial frontier model(yxuv-SAR)                     Number of obs =     6,000
                                                     Wald chi2(2)  = 118937.24
Log likelihood = -1727.016                           Prob > chi2   =    0.0000
{\smallskip}
\HLI{13}{\TOPT}\HLI{64}
           y {\VBAR} Coefficient  Std. err.      z    P>|z|     [95\% conf. interval]
\HLI{13}{\PLUS}\HLI{64}
frontier     {\VBAR}
           x {\VBAR}   1.993915   .0065251   305.58   0.000     1.981126    2.006704
         W_x {\VBAR}   .4435823   .0373189    11.89   0.000     .3704386     .516726
\HLI{13}{\PLUS}\HLI{64}
uhet         {\VBAR}
           z {\VBAR}   2.000371   .0013412  1491.49   0.000     1.997742    2.002999
\HLI{13}{\PLUS}\HLI{64}
 /lnsigma2_u {\VBAR}  -2.098104   .3163094    -6.63   0.000    -2.718059   -1.478149
 /lnsigma2_v {\VBAR}  -1.637609    .018401   -89.00   0.000    -1.673674   -1.601544
\HLI{13}{\PLUS}\HLI{64}
Wy           {\VBAR}
       _cons {\VBAR}   .6605993   .0317043    20.84   0.000     .5984599    .7227386
\HLI{13}{\PLUS}\HLI{64}
Wu           {\VBAR}
       _cons {\VBAR}   .5806681   .0318346    18.24   0.000     .5182735    .6430627
\HLI{13}{\PLUS}\HLI{64}
Wv           {\VBAR}
       _cons {\VBAR}   .5745429    .051903    11.07   0.000     .4728148     .676271
\HLI{13}{\PLUS}\HLI{64}
    sigma2_u {\VBAR}   .1226888   .0388076     3.16   0.002     .0660027    .2280593
    sigma2_v {\VBAR}   .1944444    .003578    54.34   0.000     .1875567    .2015851
         rho {\VBAR}   .3187581   .0142397    22.39   0.000     .2905787    .3463849
         tau {\VBAR}    .282414    .014646    19.28   0.000     .2534626    .3108598
       gamma {\VBAR}   .2795936     .02392    11.69   0.000     .2320763    .3257792
\HLI{13}{\BOTT}\HLI{64}

\end{stlog}

The output shows that the command fits seven equations with {\tt ml model}. The frontier equation has two explanatory variables $X_{it}$ and $W_iX_{.t}$. The scaling function uhet() has one explanatory variable $Z_{it}$.  Two equations ( /lnsigma2\_u and /lnsigma2\_v) are constructed for the variance parameters $\sigma_u^2$ and $\sigma_v^2$ which are transformed by the function $exp(\cdot)$. Three Equations (Wy, Wu, and Wv) handle the spatial dependence parameters $\rho$, $\tau$, and $\gamma$, which are parameterized as Eq.\eqref{para}. We directly include the spatial Durbin term $W_iX_{.t}$ in the frontier equation  (represented as W\_x) such that we do not need to fit a separate equation.  The bottom of the table reports the transformed parameters in the original metric.

We consider the restricted model $uv$-SAR with time-varying spatial weight matrices as the second example. The DGP 2 is described as

\begin{equation}\label{dgp2}
	Y_{it} = 1+2X_{it} + \tilde{v}_{it}-\tilde{u}_{it}, i=1,..,300; t=1,..,10
\end{equation}
where the other parameters are set the same as the DGP 1 except for $W_{i}^{ut}=W_{i}^{vt}=W_{i}^t$. The following syntax estimates the model alongside the results.

\begin{stlog}
	. use spxtsfa_DGP2.dta
{\smallskip}
. xtset id t 
{\smallskip}
Panel variable: id (strongly balanced)
 Time variable: t, 1 to 10
         Delta: 1 unit
{\smallskip}
. * importing spatial weight matrices from spxtsfa_wmat2.mmat
. mata mata matuse spxtsfa_wmat2.mmat,replace
(loading w1[300,300], w10[300,300], w2[300,300], w3[300,300], w4[300,300],
 w5[300,300], w6[300,300], w7[300,300],  w8[300,300], w9[300,300])
{\smallskip}
. * fitting the model
. local w w1 w2 w3 w4 w5 w6 w7 w8 w9 w10
{\smallskip}
. spxtsfa y x, uhet(z) wu(`w',mata) wv(`w',mata) te(efficiency) nolog
{\smallskip}
Spatial frontier model(uv-SAR)                        Number of obs =    3,000
                                                      Wald chi2(1)  = 43686.91
Log likelihood = -1336.482                            Prob > chi2   =   0.0000
{\smallskip}
\HLI{13}{\TOPT}\HLI{64}
           y {\VBAR} Coefficient  Std. err.      z    P>|z|     [95\% conf. interval]
\HLI{13}{\PLUS}\HLI{64}
frontier     {\VBAR}
           x {\VBAR}   2.015288   .0096419   209.01   0.000      1.99639    2.034186
       _cons {\VBAR}   .9415143   .0160786    58.56   0.000     .9100008    .9730278
\HLI{13}{\PLUS}\HLI{64}
uhet         {\VBAR}
           z {\VBAR}   2.000242   .0020671   967.66   0.000      1.99619    2.004293
\HLI{13}{\PLUS}\HLI{64}
 /lnsigma2_u {\VBAR}  -2.006684   .4473506    -4.49   0.000    -2.883475   -1.129893
 /lnsigma2_v {\VBAR}  -1.300024   .0260099   -49.98   0.000    -1.351002   -1.249045
\HLI{13}{\PLUS}\HLI{64}
Wu           {\VBAR}
       _cons {\VBAR}    .582383   .0031549   184.59   0.000     .5761995    .5885666
\HLI{13}{\PLUS}\HLI{64}
Wv           {\VBAR}
       _cons {\VBAR}   .5374655   .0601775     8.93   0.000     .4195198    .6554113
\HLI{13}{\PLUS}\HLI{64}
    sigma2_u {\VBAR}   .1344337    .060139     2.24   0.025       .05594    .3230678
    sigma2_v {\VBAR}   .2725253   .0070883    38.45   0.000     .2589806    .2867784
         tau {\VBAR}   .2832028   .0014508   195.21   0.000     .2803569    .2860438
       gamma {\VBAR}    .262419   .0280135     9.37   0.000      .206716    .3164261
\HLI{13}{\BOTT}\HLI{64}

\end{stlog}

In the second example, we use option {\tt te(efficiency)} to store the estimated efficiency score in a new variable {\tt efficiency}.  To show the usage of the \textit{delmissin}g option, we replace the first observation of $Y_{it}$ with missing value and re-run the above codes which gives rise to error information "\textit{missing values found. use delmissing to remove the units from the spmatrix}".  The inclusion of the \textit{delmissing} option addresses this issue and the generated variable \_\_e\_sample\_\_ records the regression sample. 

\begin{stlog}
	. * replace the first observation of y with missing value
. replace y=. in 1
(1 real changes made)
{\smallskip}
. local w w1 w2 w3 w4 w5 w6 w7 w8 w9 w10
{\smallskip}
. * estimation is aborted
. cap noi spxtsfa y x, uhet(z) wu(`w',mata) wv(`w',mata)  nolog
missing values found. use delmissing to remove the units from the spmatrix
invalid syntax
{\smallskip}

\end{stlog}

\begin{stlog}
	. * re-estimation with delmissing option 
. local w w1 w2 w3 w4 w5 w6 w7 w8 w9 w10
{\smallskip}
. spxtsfa y x, uhet(z) wu(`w',mata) wv(`w',mata) delmissing nolog
missing values found. The corresponding units are deleted from the spmatrix
{\smallskip}
{\smallskip}
Spatial frontier model(uv-SAR)                        Number of obs =    2,999
                                                      Wald chi2(1)  = 43688.15
Log likelihood = -1336.2158                           Prob > chi2   =   0.0000
{\smallskip}
\HLI{13}{\TOPT}\HLI{64}
           y {\VBAR} Coefficient  Std. err.      z    P>|z|     [95\% conf. interval]
\HLI{13}{\PLUS}\HLI{64}
frontier     {\VBAR}
           x {\VBAR}    2.01521   .0096414   209.02   0.000     1.996313    2.034106
       _cons {\VBAR}   .9409358   .0160983    58.45   0.000     .9093837     .972488
\HLI{13}{\PLUS}\HLI{64}
uhet         {\VBAR}
           z {\VBAR}   2.000244   .0020675   967.48   0.000     1.996192    2.004296
\HLI{13}{\PLUS}\HLI{64}
 /lnsigma2_u {\VBAR}  -2.006689   .4473506    -4.49   0.000     -2.88348   -1.129898
 /lnsigma2_v {\VBAR}  -1.300032   .0260157   -49.97   0.000    -1.351022   -1.249042
\HLI{13}{\PLUS}\HLI{64}
Wu           {\VBAR}
       _cons {\VBAR}   .5823337   .0031574   184.44   0.000     .5761453     .588522
\HLI{13}{\PLUS}\HLI{64}
Wv           {\VBAR}
       _cons {\VBAR}   .5400995   .0602196     8.97   0.000     .4220712    .6581278
\HLI{13}{\PLUS}\HLI{64}
    sigma2_u {\VBAR}    .134433   .0601387     2.24   0.025     .0559397    .3230661
    sigma2_v {\VBAR}   .2725231   .0070899    38.44   0.000     .2589755    .2867794
         tau {\VBAR}   .2831801   .0014519   195.04   0.000     .2803319    .2860233
       gamma {\VBAR}   .2636448   .0280137     9.41   0.000     .2079367    .3176476
\HLI{13}{\BOTT}\HLI{64}
Missing values found
The regression sample recorded by variable __e_sample__

\end{stlog}

Finally, we consider another restricted model $xuv$-SAR with different spatial weight matrices, one of which is time-varying, and the others are time-constant.  The model is described as DGP 3:
\begin{equation}\label{dgp3}
	Y_{it} = 1+2X_{it}+ 0.5W_{i}^{xt} + \tilde{v}_{it}+\tilde{u}_{it}, i=1,..,300; t=1,..,10
\end{equation}
where the other parameters are set the same as the DGP 1 except for $W_{i}^{ut}=W_{i}^u$ and $W_{i}^{vt}=W_{i}^v$.  Different from DGP 1 and DGP 2, which set the production function frontier, DGP 3 specifies a cost function. The estimation of the model is shown as follows.

\begin{stlog}
	. use spxtsfa_DGP3.dta
{\smallskip}
. xtset id t 
{\smallskip}
Panel variable: id (strongly balanced)
 Time variable: t, 1 to 10
         Delta: 1 unit
{\smallskip}
. * importing spatial weight matrices from spxtsfa_wmat2.mmat
. mata mata matuse spxtsfa_wmat2.mmat,replace
(loading w1[300,300], w10[300,300], w2[300,300], w3[300,300], w4[300,300], w5[300,300], w6[300,300], w7[300,300],
 w8[300,300], w9[300,300])
{\smallskip}
. * fitting the model
. local w w1 w2 w3 w4 w5 w6 w7 w8 w9 w10
{\smallskip}
. mat b = (1,1,1,1,-1,-1,0.5,0.5)
{\smallskip}
. spxtsfa y x, uhet(z) wu(w2,mata) wv(w1,mata) wxvars(x) ///
>              wx(`w',mata) cost init(b) genwxvars nolog
{\smallskip}
Spatial frontier model(xuv-SAR)                       Number of obs =    3,000
                                                      Wald chi2(2)  = 57430.99
Log likelihood = -872.06794                           Prob > chi2   =   0.0000
{\smallskip}
\HLI{13}{\TOPT}\HLI{64}
           y {\VBAR} Coefficient  Std. err.      z    P>|z|     [95\% conf. interval]
\HLI{13}{\PLUS}\HLI{64}
frontier     {\VBAR}
           x {\VBAR}   1.995734   .0083857   237.99   0.000     1.979298    2.012169
         W_x {\VBAR}   .5065867   .0221169    22.90   0.000     .4632384     .549935
       _cons {\VBAR}   .9916378   .0126993    78.09   0.000     .9667476    1.016528
\HLI{13}{\PLUS}\HLI{64}
uhet         {\VBAR}
           z {\VBAR}    1.99976   .0010759  1858.74   0.000     1.997652    2.001869
\HLI{13}{\PLUS}\HLI{64}
 /lnsigma2_u {\VBAR}  -1.699017   .4472788    -3.80   0.000    -2.575668    -.822367
 /lnsigma2_v {\VBAR}  -1.615448   .0260424   -62.03   0.000     -1.66649   -1.564406
\HLI{13}{\PLUS}\HLI{64}
Wu           {\VBAR}
       _cons {\VBAR}   .6214083   .0008566   725.44   0.000     .6197294    .6230872
\HLI{13}{\PLUS}\HLI{64}
Wv           {\VBAR}
       _cons {\VBAR}   .6001869   .0595911    10.07   0.000     .4833905    .7169833
\HLI{13}{\PLUS}\HLI{64}
    sigma2_u {\VBAR}   .1828631   .0817908     2.24   0.025      .076103    .4393904
    sigma2_v {\VBAR}   .1988017   .0051773    38.40   0.000      .188909    .2092123
         tau {\VBAR}   .3010474   .0003894   773.04   0.000      .300284    .3018105
       gamma {\VBAR}    .291369   .0272628    10.69   0.000     .2370726    .3438504
\HLI{13}{\BOTT}\HLI{64}

\end{stlog}

In the third example, we use {\tt cost} option to specify the type of frontier.  The matrix {\tt b} is used as the initial value for the maximum likelihood estimation. The likelihood function of spatial stochastic frontier models is complicated, and generally difficult to obtain the optimal global solutions. Thus, good initial values would be helpful for fitting spatial stochastic models. Practitioners might fit the non-spatial stochastic models using  {\tt fronteir} and {\tt sfpanel} commands to obtain the initial values of the parameters involved in the frontier and the scaling function and then use the {\tt mlplot} option to search initial values for spatially-correlated parameters.
 
%\section{Empirical applications}

\section{Conclusion}\label{sec_conclusion}

Geospatial units are not isolated or separated but connected. For example, the economic trade, social activities, and cultural exchange between different regions affect each other. Such spatial interdependence challenges the traditional econometric methods, which generally assume cross-sectional independence. Spatial econometrics is developed to handle spatial correlation. Recently, researchers combined stochastic frontier models with spatial econometrics to account for various types of spatial effects in the field of efficiency and productivity analysis \citep{galli2022spatial,orea2019new}.  This article presented a community-contributed command for fitting spatial stochastic frontier models with different sources of spatial dependence. We hope the developed command can provide some convenience to practitioners and reduce the difficulty of model applications, thereby promoting sound empirical research. 
%Finally, there are some limitations that should be duly noted. First, the spatial stochastic frontier models require prior information on the spatial weight matrices. Second, the distribution of the inefficiency is assumed to be half-normal. These settings might affect the estimated results.




\section{Acknowledgments}
Kerui Du thanks the financial support of the National Natural Science Foundation of China (72074184).  We are grateful to Federica Galli for his Matlab codes, Federico Belotti, Silvio Daidone, Giuseppe Ilardi and Vincenzo Atella for the sfcross/sfpanel package, Mustafa U. Karakaplan for the sfkk package, and Jan Ditzen, William Grieser and Morad Zekhnini for the nwxtregress package which inspired our design of the {\tt spxtsfa} command. 



\endinput



\bibliographystyle{sj}
\bibliography{sj}


\begin{aboutauthors}

Kerui Du is an associate professor at the School of Management, Xiamen University. His primary research interests include applied econometrics, energy and environmental economics.	

Luis Orea is a professor at the School of Economics and Business, University of Oviedo. His primary research interests include Efficiency and productivity analysis, econometric modelling, agricultural economics, energy economics, regulation and competition, spatial economics. 

Inmaculada C. Álvarez is a professor at the Department of Economics, Universidad Autónoma de Madrid. Her primary research interest include infrastructures, efficiency and productivity, economic growth and development, spatial economics and quantitative methods. 
	

	
\end{aboutauthors}


\endinput
